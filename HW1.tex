\documentclass{article}
\usepackage[utf8]{inputenc}
\usepackage[english]{babel}
\usepackage[]{amsthm,enumitem,amsmath,eurosym} %%[label=(\alph*)] for lettered list
\usepackage{geometry}
\geometry{a4paper, total={6in, 8in}}
\usepackage[]{amssymb}
\newcommand{\modd}{\text{mod }}
\newcommand{\Z}{\mathbb{Z}}
\newcommand{\R}{\mathbb{R}}
\newcommand{\Q}{\mathbb{Q}}
\newcommand{\N}{\mathbb{N}}
\newcommand\power[1]{\mathcal{P}(#1)}


\title{Homework \#1}
\author{Tobias Dault}
\date{\today}


\begin{document}
	\maketitle 
	
	\section*{Section 1.1}
	\subsection*{Problem 2}
	Let $S=\{-2,-1,0,1,2,3\}$. Describe each of the following sets as $\{x\in S : p(x)\}$, where $p(x)$ is some condition on $x$.
	\begin{enumerate}[label=(\alph*)]
		\item $A=\{1,2,3\}=\{x\in S: x\in\mathbb{N}\}$
		\item $A=\{0,1,2,3\}=\{x\in S: x\geq0\}$
		\item $A=\{-2,-1\}=\{x\in S: x<0\}$
		\item $A=\{-2,2,3\}=\{x\in S: |x|>1\}$
	\end{enumerate}
	\subsection*{Problem 4}
	\begin{enumerate}[label=(\alph*)]
		\item $A=\{n\in\mathbb{Z}:-4<n\leq4\}=\{-3,-2,-1,0,1,2,3,4\}$
		\item $B=\{n\in\mathbb{Z}:n^2<5\}=\{-2,-1,0,1,2\}$
		\item $C=\{n\in\mathbb{N}:n^3<100\}=\{1,2,3,4\}$
		\item $D=\{n\in\mathbb{R}:x^2-x=0\}=\{0,1\}$
		\item $E=\{n\in\mathbb{R}:x^2+1=0\}=\emptyset$
	\end{enumerate}
	\subsection*{Problem 8}
	Let $A=\{n\in\Z:2\leq|n|<4\}$, $B=\{x\in\Q:2<x\leq4\}$, $C=\{x\in\R:x^2-(2+\sqrt{2})x+2\sqrt{2}=0\}$, and $D=\{x\in\Q:x^2-(2+\sqrt{2})x+2\sqrt{2}=0\}$.
	\begin{enumerate}[label=(\alph*)]
		\item $A=\{-3,-2,2,3\}$
		\item $\{\frac{5}{2},\frac{7}{2}, 4\}\in B$, $\{\frac{5}{2},\frac{7}{2}, 4\}\notin A$
		\item $C=\{\sqrt{2},2\}$
		\item $D=\{x\in\Q:(x-2)^2(x-\sqrt{2})^2=0\}$
		\item $|A|=4, |C|=2, |D|=1$
	\end{enumerate}

	
	\section*{1.2}
	\subsection*{Problem 10}
	Give examples of three sets $A$, $B$, and $C$ such that
	\begin{enumerate}[label=(\alph*)]
		\item $A\subseteq B\subset C$\\
		$A=\{1\}$\\
		$B=\{1\}$\\
		$C=\{1,2\}$
		\item $A\in B, B\in C$ and $A\notin C$\\
		$A=\{1\}$\\
		$B=\{\{1\}\}$\\
		$C=\{\{\{1\}\}\}$
		\item $A\in B$ and $A\subset C$\\
		$A=\{1\}$\\
		$B=\{\{1\}\}$\\
		$C=\{1,2\}$
	\end{enumerate}

	\subsection*{Problem 12}
	Which of the follow sets are equal?
	\begin{enumerate}[label=(\alph*)]
		\item $A=\{n\in Z: |n|<2\}$
		\item $B=\{n\in Z: n^3=n\}$
		\item $C=\{n\in Z: n^2\leq n\}$
		\item $D=\{n\in Z: n^2\leq1\}$
		\item $E=\{-1,0,1\}$
	\end{enumerate}
	$A,B,D,$ and $E$ are equal.
	\subsection*{Problem 14}
	Find $\mathcal{P}(A)$ and $|\mathcal{P}(A)| for$
	\begin{enumerate}[label=(\alph*)]
		\item $A=\{1,2\}$\\
		$\mathcal{P}(A)=\{\emptyset,\{1\},\{2\},\{1,2\}\}$\\
		$|\mathcal{P}(A)|=4$
		\item $A=\{\emptyset,1,\{a\}\}$\\
		$\mathcal{P}(A)=\{\emptyset,\{\emptyset\},\{1\},\{\{a\}\},\{\emptyset,1\},\{\emptyset,\{a\}\},\{1,\{a\}\},\{\emptyset,1,\{a\}\}\}$\\
		$|\mathcal{P}(A)|=8$
	\end{enumerate}
	\subsection*{Problem 16}
	Find $\mathcal{P}(\mathcal{P}(A))$ and its cardinality.\\
	$\mathcal{P}(A)=\{\emptyset,\{\emptyset\},\{\{1\}\},\{\emptyset, 1\}\}$
	\subsection*{Problem 20}
	Determine whether the following statements are true or false.
	\begin{enumerate}[label=(\alph*)]
		\item If $\{1\}\in\power{A}$, then $1\in A$ but $\{1\}\notin A$. FALSE
		\item If $A,B$ and $C$ are sets such that $A\subset\power{B}\subset C$ and $|A|=2$, then $|C|$ can be 5 but $|C|$ cannot be 4. TRUE
		\item If a set $B$ has one more element than a set $A$, then $\power{B}$ has at least two more elements than $\power{A}$. FALSE
		\item If four sets $A,B,C,$ and $D$ are subsets of $\{1,2,3\}$ such that $|A|=|B|=|C|=|D|=2$, then at least two of these sets are equal. TRUE
	\end{enumerate}
	
	\section*{1.3}
	\subsection*{Problem 22}
	Let $U=\{1,3,\dotsc,15\}$ be the universal set, $A=\{1,5,9,13\}$, and $B=\{3,9,15\}$. Determine the following:
	\begin{enumerate}[label=(\alph*)]
		\item $A\cup B=\{1,3,5,9,13,15\}$
		\item $A\cap B=\{9\}$
		\item $A-B=\{1,5,13\}$
		\item $B-A=\{3,15\}$
		\item $\bar{A}=\{3,7,11,15\}$
		\item $A\cup\bar{B}=\{1,5,7,9,11,13\}$
	\end{enumerate} 
	\subsection*{Problem 24}
	Give examples of three sets $A,B$ and $C$ such that $B\neq C$ but $B-A=C-A$.\\
	$A=\{1\}$\\
	$B=\{1\}$\\
	$C=\{1,1\}$
	\subsection*{Problem 30}
	Let $A=\{x\in\R:|x-1|\leq2\},B+\{x\in\R:|x|\geq1\}$ and $C=\{x\in\R:|x+2|\leq3\}$.
	\begin{enumerate}[label=(\alph*)]
		\item Express $A,B,$ and $C$ using interval notation.\\
		$A=[-1,3]$\\
		$B=(-\infty,1]\cup[1,\infty)$\\
		$C=[-5,1]$
		\item Determine each of the following sets using interval notation:\\
		\begin{enumerate}[label=(\arabic*)]
			\item $A\cup B=(-\infty,\infty)$\\
			\item $A\cap B=[-1,1]\cup[1,3]$
			\item $B\cup C=(-\infty,\infty)$
			\item $B-C=(-\infty,-5)\cup(1,\infty)$
		\end{enumerate}
	\end{enumerate}
	
	\section*{1.4}
	\subsection*{Problem 1.36}
	For a real number $r$, define $S_r$ to be the interval $[r-1,r+2]$. Let $A=\{1,3,4\}$. Determine $\bigcup_{\alpha\in A}S_\alpha$ and $\bigcap_{\alpha\in A}S_\alpha$.\\
	$\bigcup_{\alpha\in A}S_\alpha=[0,3]\cup[2,5]\cup[3,6]=[0,6]$\\
	$\bigcap_{\alpha\in A}S_\alpha=[0,3]\cap[2,5]\cap[3,6]=[3]$
	\subsection*{Problem 1.38}
	For a real number $r$, define $A_r=\{r^2\},B_r$ as the closed interval $[r-1,r+1]$ and $C_r$ as then interval $(r,\infty)$. For $S=\{1,2,4\}$.
	\begin{enumerate}[label=(\alph*)]
		\item 
		$\bigcup_{\alpha\in S}A_\alpha=\{1,4,16\}$\\
		$\bigcap_{\alpha\in S}A_\alpha=\emptyset$
		
		\item
		$\bigcup_{\alpha\in S}B_\alpha=[0,2]\cup[1,3]\cup[3,5]=[0,5]$\\
		$\bigcap_{\alpha\in S}B_\alpha=[0,2]\cap[1,3]\cap[3,5]=\emptyset$
		\item
		$\bigcup_{\alpha\in S}C_\alpha=(1,\infty)\cup(2,\infty)\cup(4,\infty)=(1,\infty)$\\
		$\bigcap_{\alpha\in S}C_\alpha=(1,\infty)\cap(2,\infty)\cap(4,\infty)=(4,\infty)$
	\end{enumerate}
	\subsection*{Problem 1.40}
	For $i\in\Z$, let $A_i=\{i-1,i+1\}$. Determine the following:
	\begin{enumerate}[label=(\alph*)]
		\item $\bigcup\limits_{i=1}^{5} A_{2i}=\{1,3\}\cup\{3,5\}\cup\{5,7\}\cup\{7,9\}\cup\{9,11\}=\{0,1,3,5,7,9,11\}$
		\item $\bigcup\limits_{i=1}^{5} (A_{i}\cap A_{i+1})=$
		\item $\bigcup\limits_{i=1}^{5} A_{2i-1}\cap A_{2i+1}$
		
	\end{enumerate}
	\subsection*{Problem 1.44}
	Each of the following sets is a subset of $A=\{1,2,\dotsc,10\}$:\\
	

	Find a set $I\subseteq\{1,2,\dotsc,13\}$ such that for every two distinct elements $j,k\in I, A_j\cap A_k=\emptyset$ and $|\bigcup_{i\in I}A_i|$ is maximum.
	
	
\end{document}
